\documentclass[11pt]{article}
\usepackage{CJK}
\usepackage{verbatim}
\usepackage{graphpap}
\usepackage{fancyvrb}
\usepackage{amssymb}
\usepackage{amsmath}
\usepackage{enumerate}
\begin{document}
\begin{CJK}{UTF8}{gbsn}
\author{舒何}
\title{高等代数}
\maketitle
\begin{center}
	中文你知道\\[1mm]
	{\large 高等代数}\\[4mm]
	{\small 作者:舒何}
\end{center}
\begin{quotation}
黑洞是恒星的一种残骸,它是引力收缩第极点,极端道近乎荒唐。\par
\raggedleft——约翰$\cdot$ 卢米涅	
\end{quotation}
\begin{CJKverbatim}
\begin{verbatim}
	文本
\end{verbatim}
\begin{equation}
	x_1+x_2+x_3+\dots+x_n=y
\end{equation}
\end{CJKverbatim}
To show the effect of the matrix on surrounding lines insides a paragraph ,we put it here:
\begin{equation}\nonumber
\begin{pmatrix}
	\begin{smallmatrix}
		1 & 0\\
		0 & -1
	\end{smallmatrix}
\end{pmatrix}
\end{equation}
\pageref{equ:gongshi}.\ref{equ:zhengming} and follow it with enough text to ensure that there is at least one full line below the matrix\\
$\underbrace{a+b+c+\dots+z}_{26}$
shelfful\\
shelf\mbox{}fule\\
\begin{tabular}{|l|l|l|}
\hline
dsaw & hello & shuhe\\
nishuo& wanan & \\ 
\cline{2-3}
unuse & yishi &\\
\hline
\hline 12 & decimal &\\
\hline	
\end{tabular}
\section{第一章}
\subsection{求和公式}
\begin{displaymath}
	\lim_{n \to \infty}
	\sum_{k=1}^n \frac{1}{k^2}=\frac{\pi^2}{6}
\end{displaymath}
\begin{equation}\label{equ:zhengming}
	\forall x\in \mathbb{R}: \qquad x^2\geq 0
\end{equation}
$e^{x^2}\neq e^{x2}$\qquad
$a_{i,j}^3$\qquad $\sqrt{x^2+\sqrt{y}}$\qquad$\underline{m+n}\quad\overline{m+n}\qquad$\\
\begin{displaymath}
	y=x^2\qquad y'=2x\qquad y''=2
\end{displaymath}
\begin{displaymath}
	\vec A\qquad\overrightarrow{AB}
\end{displaymath}\\
\[\lim_{x \rightarrow 0}\frac{sinx}{x}
\]
\[v=\sigma_1\cdot\sigma_2\tau_1\cdot\tau_2 
\]
\begin{displaymath}
\begin{split}
	{n \choose k}\\[4pt]
\frac{x^2}{k+1}\label{equ:gongshi}\\
1+\left(\frac{1}{1-x^2}\right)^3\\
\int_a^b{x^2dx}\\
\sum_{i=1}^{\frac{5}{y}}{1+2+\cdots+n=y_n}\\
1+\left(2+3..+n )\right.\\
\Big((x+1)(x-1)\Big)^2\\
\mathbf{X}=
\left(\begin{array}{c|c|c}
x_{11} & x_{12} & \ldots\\
x_{21} & x_{22} & \ldots\\
\vdots & \vdots & \ddots \end{array}\right)\\
\mathop{\mathrm{corr}}{X,Y}=
\frac{\displaystyle\sum_{i=1}^n(x_i-\overline x)(y_i-\overline y)}{\displaystyle\biggl[ \sum_{i=1}^n(x_i-\overline x)^2\sum_{i=1}^n(y_i-\overline y)^2\biggr]^{1/2}}\\
\end{split}
\end{displaymath}
\begin{eqnarray}
	\label{equ:equo}(1+\tau\delta_x)u_{ij}^* & = & u_{ij}^{n-1}+\tau f\\
	(1+\tau\delta_y)u_{ij}^n & = & u_{ij}^*
\end{eqnarray}
对于3.\ref{equ:equo}\,式来说,可以改写成矩阵
\begin{equation*}
	Au_{ij}^*=u_{ij}^{n-1}+\tau f
\end{equation*}
即
\newcommand{\gal}{g_\alpha^}
\begin{align*}
	A=({K_x}{r_x})(C+C^T)+I\\
	C=\left( \begin{array}{ccccc}
		\gal1 & \gal0 & 0 & 0 & 0\\
		\gal2 & \gal1 & \gal0 & 0 & 0\\
		\gal3 & \gal2 & \gal1 & \gal0 & 0\\
		\vdots & \vdots &\ddots &\gal1 & \gal0\\
		\gal{m_1-1}&\gal{m_1-2} &\ldots & \gal2 & \gal1
\end{array}\right)\\
\infty \int_a^b\sin xdx=-\cos x\arrowvert _a^b\\
\end{align*}
\begin{displaymath}
\lim_{x\rightarrow0}\frac{\sin x}{x}=\left\{ \begin{array}{ll}
	a & ss\\
	b & cc\\
	c & hh\\
	d & abdc\\
	you &know\\\
\end{array} \right.
\end{displaymath}
\section{theroem}
\subsection{subtheroem}
hello
\newtheorem{dingli}{定理}[subsection]
\begin{dingli}
如果有一个人是真的
\end{dingli}
\begin{dingli}
还是一个定理
\end{dingli}
\newtheorem{hello}{引理}[section]
\begin{hello}
	那~么他就是真的
\end{hello}
\begin{enumerate}[(1)]
	\item latex1
	\item latex2
\end{enumerate}
%hello\verb|\ve.     rb| zhuzidayin 
\newcommand{\pd}{\partial}
\newcommand{\fl}[2]{\frac{\pd^{\,#2}u}{\pd{\vert #1 \vert}^{#2}}}
\numberwithin{equation}{section}
\begin{eqnarray}
	K_x\fl{x}{\alpha}+K_y\fl{y}{\beta}+f(u,x,y,t)\\	
\fl{x}{\alpha}=-c_{\alpha}\left(\fl{x}{\alpha}+\fl{\!-\!x}{\alpha} \right)\\
\fl{y}{\beta}=-c_{\beta}\left(\fl{y}{\beta}+\fl{\!-\!y}{\beta}\right)
\end{eqnarray}
\section{第三章}{这是第三章}
\begin{gather}
	\fl{y}{\beta}
\end{gather}
\subsection{小节}
\begin{multline}
		\fl{y}{\beta}
\end{multline}
\end{CJK}
\end{document}